% Created 2020-04-21 Tue 18:33
% Intended LaTeX compiler: pdflatex
\documentclass[DIV=13,halfparskip,11pt,headinclude]{scrartcl}
\usepackage[utf8]{inputenc}
\newcommand{\un}[1]{\ \textrm{#1}}
\newcommand{\rd}{\,\mathrm{d}}
\newcommand{\pd}[2]{\frac{\partial {#1}}{\partial {#2}}}
\newcommand{\Div}[1]{\nabla \cdot \mathbf{#1}}
\newcommand{\bm}[1]{\mathbf{#1}}
\usepackage[utf8]{inputenc}
\usepackage[T1]{fontenc}
\usepackage{graphicx}
\usepackage{grffile}
\usepackage{longtable}
\usepackage{wrapfig}
\usepackage{rotating}
\usepackage[normalem]{ulem}
\usepackage{amsmath}
\usepackage{textcomp}
\usepackage{amssymb}
\usepackage{capt-of}
\usepackage{hyperref}

\author{Mauro A Werder}
\date{\today}
\title{}
\hypersetup{
 pdfauthor={Mauro A Werder},
 pdftitle={},
 pdfkeywords={},
 pdfsubject={},
 pdfcreator={Emacs 26.3 (Org mode 9.1.9)},
 pdflang={English}}
\begin{document}

\section{Flow routing}
\label{sec:org5e20d4b}

For flow routing we're after the steady-state water flux \(\mathbf{q}=D
\mathbf{u}\) at each location (or more like \(q=|\mathbf{q}|\)) for given bed elevation \(z_b\), ice
thickness \(H\), and source \(M\).  We actually don't care about the water
layer thickness \(D\), except maybe where it is large, i.e. where there
are lakes.

The flux obeys the conservation equation
\begin{equation}
\frac{\partial D}{\partial t} = M - \nabla \cdot (\mathbf{u} D)
\end{equation}
with \(D\) water layer depth, \(M\) source, and \(\mathbf{u}\) depth-averaged flow
velocity.

\(\mathbf{u}\) is some function of the hydraulic potential gradient
\begin{equation}
 \mathbf{u} = f(\nabla \phi)
\end{equation}
usually something like Darcy-Weisbach
\begin{equation}
 \mathbf{u} = -k D^{\alpha-1}\frac{\nabla \phi}{\sqrt{|\nabla
\phi|}},
\end{equation}
where \(k\) is a conductivity. Or just linear
\begin{equation}
 \mathbf{u} = -k \nabla \phi.
\end{equation}

But maybe to just get the
steady state water flux, we'll use something else to get, e.g., a good return
on the CFL condition. (Also, \(\mathbf{u}\) might be adjusted to also take
deflections due to bed-slope and pressure melting point effects into account.)

The hydraulic potential is
\begin{equation}
  \label{eq:phi}
\phi = \rho_w g z_b + p_w
\end{equation}
where the water pressure \(p_w\) can be approximated by assuming it
equal to ice overburden pressure
\begin{equation}
  \label{eq:shreve}
  p_w= p_i = \rho_i g H,
\end{equation}
where \(H\) is ice thickness.
Adding also the contribution of the water
sheet to allow flooding of depressions gives
\begin{equation}
  p_w =  \rho_w g D + \rho_i g H.
\end{equation}
Note that this is usually
called the "diffusion wave approximation".

Without the \(D\) term in \(\phi\), we get the "kinematic wave"
\begin{equation}
\phi^* = \rho_w g z_b + \rho_i g H,
\end{equation}
But now water will keep accumulating in local depressions.  Thus this
is best suited to filled DEMs.

The aim is to recover \(q\) by running Eq.1 to steady-state or to
directly solve \begin{equation}
0 = M - \nabla \cdot (\mathbf{u} D).
\end{equation}

\subsection{Keeping the surface fixed}

Instead of formulating eq.~\eqref{eq:shreve} in terms of
ice-thickness, it can be formulated in terms of a fixed surface
elevation
\begin{equation}
  p_w = p_i + p_D = \rho_i g (z_s - z_b - D) + \rho_w g D,
\end{equation}
where the first term is the ice thickness and the second is the
contribution of $D$.  This is potentially the more natural approach as
in the usual setting the surface is given.  This will mean, as $D$
grows it will replace ice with water.

Inserting into eq.~\eqref{eq:phi} and rearranging gives
\begin{equation}
\phi = \rho_w g \left( \frac{\rho_i}{\rho_w}z_s + \left(1- \frac{\rho_i}{\rho_w}\right)( z_b + D)\right).
\end{equation}

As a side note, probably the surface elevation should be smoothed over
a length scale of around one ice thickness, as smaller surface
features will not impact the pressure at the bed.

\subsection{Boundary conditions on $D$}


No flow boundary condition is setting $D=0$ as then $q=0$.

Free flow BC is when $\frac{\partial q}{\partial n}=0$, where $n$ is
the boundary normal.

\subsection{Scaled equations}
The system
\begin{align}
  \frac{\partial D}{\partial t} = M - \nabla \cdot (\mathbf{u} D)\\
  \phi = \rho_w g (z_b + D) + \rho_i g H\\
  \mathbf{u} = -k \nabla \phi
\end{align}
can be scaled (using $u = \hat u \bar u$, etc., and dropping the bar
immediately) to
\begin{align}
  \frac{\partial D}{\partial t} = M - \nabla \cdot (\mathbf{u} D)\\
  \phi = z_b + r_{Dz}D + r_\rho H\\
  \mathbf{u} = - k \nabla \phi
\end{align}
where all variables are now dim-less, and the parameters $r_{Dz} =
\frac{\hat{D}}{\hat z}$, $r_\rho=\frac{\hat \rho_i}{\hat \rho_w}$,
and $k$ is the unscaled $k$ divided by $\hat k$.  With
scales
\begin{center}
      \renewcommand{\arraystretch}{1.2}
  \begin{tabular}{lllll}
Scaling var & def & val & units & remarks\\
\hline
  $\hat g$ & set to & 9.81 & m/s \\
  $\hat \rho_w$  & set to & 1000 & kg/m$^3$ \\
  $\hat \rho_i$  & set to & 910 & kg/m$^3$ \\
  $\hat M$ & set to & 10$^{-6}$ & m/s & 1mm/day$\sim10^{-5}$, 1mm/a$\sim10^{-8}$\\
  $\hat D$       & set to & 1   & m \\
  $\hat \phi$ & set to & $10^7$ & Pa\\
  $\hat x$ & set to & 10$^6$ & m\\
    \hline
    $\hat t$ & $\hat D/\hat M$ & $10^6$ & s\\
    $\hat z$ & $\hat \phi/(\hat g \hat \rho_w)$ & 1019 & m \\
    $\hat \Psi$ & $\hat \phi/\hat x$ & 10 & Pa/m\\
    $\hat u$ & $\hat x / \hat t$ & 1 & m/s\\
    $\hat k$ & $\hat u / \hat\Psi$ & 0.1 & m$^3$s/kg\\
  \hline
  $r_\rho$ & $\hat \rho_i/\hat \rho_w$ & 0.91 & -\\
  $r_{Dz}$ & $\hat D/\hat z$ & 1/1019 & -\\
\end{tabular}
\end{center}
Note that the dimension-full $k$ can be set arbitrarily (which is good
as we don't know/care about it).

Similarly,
\begin{equation}
\phi = \rho_w g \left( \frac{\rho_i}{\rho_w}z_s + \left(1- \frac{\rho_i}{\rho_w}\right)(
  z_b + D)\right)
\end{equation}
can be scaled to
\begin{equation}
\phi = r_\rho z_s + (1- r_\rho)z_b + r_{Dz}(1- r_\rho)D.
\end{equation}
The other equations scale identically.

\subsection{Reduction with \(|u|=1\) for the kinematic wave approx.}
\label{sec:orgbae1021}

As we're only interested in the steady state, we can choose to fix the
magnitude of \(\mathbf{u}\) to a constant.  What this then entails is
that \(D\) adjusts such that all the discharge accumulated from upstream
and the local source can be routed away; in fact \(D=q\).

\begin{itemize}
\item \(\mathbf{\hat u} = \frac{\nabla \phi^*}{|\nabla \phi^*|}\) can be
calculated from the DEM
\item choose \(\Delta t\) such that CFL condition is met
\item probably use some sort of upwind finite-diffs
\end{itemize}

\subsection{Reduction with \(|u|=1\) for the diffusion wave approx.}
\label{sec:org7f4e9a4}

I don't think a trick like this is possible here: with a fixed \(|u|=1\)
a lake cell with a large \(D\) would then necessarily have a high
discharge.  However, lake cells with low discharge are well possible.

However, we can use a different eq. for \(\mathbf{u}\) as long as \(u\) is
a decreasing function of \(\nabla \phi\), e.g. simply
\begin{equation}
\mathbf{u} = - \nabla \phi.
\end{equation}

\subsection{TODO}
Think about how lakes could be filled faster.  (pre-fill DEM, increase
source in lakes)

Scheme to keep surface constant and evolve $H$.

\section{Numerics}

Upwind scheme seems needed.

\section{Test cases}
\label{sec:orgb000489}
\subsection{1D synthetic test case}
\label{sec:org1b6b5f1}

See \url{https://shmip.bitbucket.io/instructions.html\#sec-2-2} for DEMs.

Set \(M=1\).

\subsection{2D synthetic test case}
\label{sec:org08a60fa}

\section{Application to Antarctica}
\label{sec:orgf1a5640}

Application needs to cover Antarctica at 500m or better resolution.
Running a domain of 6000km\(^{\text{2}}\), this results in 10\(^{\text{8}}\) grid-points for 500m and
10\(^{\text{9}}\) for 100m resolution.

This would need a few GB of memory.

\section{Refs}
\label{sec:org50e7c95}
Le Brocq et al 2009: \url{https://doi.org/10.3189/002214309790152564}

Diffusion wave eq \url{https://doi.org/10.1016/j.jhydrol.2019.123925}
\end{document}
%%% Local Variables:
%%% mode: latex
%%% TeX-master: t
%%% End:
